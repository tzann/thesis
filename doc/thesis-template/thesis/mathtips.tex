\chapter{More Tips on Math/\LaTeX{}}\label{chapter:mathtips}

\section{The basics}
\begin{Rule} [Display math] You might have heard at some point that it is not a great idea to use \lstinline|$$...$$|. Indeed,
the preferred way is to use \lstinline|\[...\]| or equivalently \lstinline|\begin{displaymath}...\end{displaymath}|.
The problem with the former is some issues with spacing and also that the command
\lstinline|\fleqn| (which flushes math to the left) does not work.     
\end{Rule}
\begin{Rule}[Punctuation] Speaking about inline math and display math, keep in mind that one should have punctuation outside of inline 
math, but inside display math. That is, we say: The random variable $X$, which has expectation 5, is positive.
But we would say: the expectation of the random variable $X$, which satisfies
\[X > 0,\]
is 5.
\end{Rule}
\begin{Rule}[Roman font] For functions like $\log{n}, \cos{n}$, don't just write their name in math, i.e. $logn, cosn$. Instead,
use the macros when available (as here), or if they are not you can use \lstinline|\mathrm{function}|, which prints 
$\mathrm{function}$. This uses the roman font in math. Note that this is also what we have done for defining the \lstinline|\Var| macro.
\end{Rule}
\begin{Rule}[Dots] Do not use \lstinline|$(x_1, ..., x_n)$|. Instead, you should use \lstinline|$(x_1, \ldots, x_n)$|, i.e. $(x_1, \ldots, x_n)$ when you 
want lowered dots and \lstinline|$ A \times \cdots \times Z$| ($ A \times \cdots \times Z$) when you want centered dots.
\end{Rule}
\section{Some common math for reference/uniformity}
\begin{Rule}[Set building]A very common thing to do is define sets via some sort of condition, like so:
\[X = \SetBuilder*{x \in \R}{f(x) > 0}.\]
You can use our macro \lstinline|\Set| for this, which takes two arguments, the left and right-hand side
of the $:$ symbol. You can also replace this with \lstinline|\mid| ($\mid$), but make sure to be consistent.    
\end{Rule}

\begin{Rule}[Cases] Sometimes we want to define a piecewise function. This can be done with the \lstinline|\cases| environment.
\[
f(n) = \begin{cases}
            n, & \text{if $n < T/2$}\\
            T - n, & \text{otherwise}
 \end{cases}\]
\end{Rule}
\begin{Rule}[Matrices] Matrices are also super common, here are some examples:
\[\begin{pmatrix}
    1 & 2 & 3\\
    a & b & c
    \end{pmatrix}, \quad 
    \begin{bmatrix}
    1 & 2 & 3\\
    e & f & g
    \end{bmatrix}, \quad
    \begin{Bmatrix}
    1 & 2 & 3\\
    h & i & j
    \end{Bmatrix}\]
One thing that you might not know is that you can have small matrices for inline math, like so: $\big(\begin{smallmatrix}
  k & l\\
  m & n
\end{smallmatrix}\big)$.
\end{Rule}
\begin{Rule}[Binomials] Finally, it is highly likely that you use binomial coefficients somewhere in your thesis, like so:
\[\binom{n}{k}.\]
\end{Rule}
\section{Figures and Algorithms}
Almost always it is a good idea to use figures to illustrate your arguments and help the 
reader understand them more deeply. For this, you can use TikZ, documented \href{https://tikz.dev/}{here}. Below we give 
an example figure (\Cref{fig:exampletikzgraph}) with some elements that might be useful. For other examples, look at the documentation (it is actually good 
and not just incomprehensible walls of text, i.e.\ it also has examples you can modify to your needs easily).


\begin{figure}[ht]
    \centering
    \begin{tikzpicture}[
        % Define common TikZ styles
        >=Stealth,              % Use Stealth arrow tips
        auto,                   % Position edge labels automatically
        node distance=2cm,      % Default distance between nodes
        thick,                  % Default line thickness
        every node/.style={font=\small},  % Node font size
        vertex/.style={
            circle, 
            draw, 
            minimum size=1.2em,
            inner sep=2pt
        },
        weight/.style={
            fill=white,
            inner sep=2pt
        }
    ]
    
    % Define vertices:
    % The 'label' key can be used to add labels outside the vertex;
    % an alternative is to put the label text directly in the node content (e.g., {A})
    \node[vertex, fill=blue!20, label=above:{$v_1$}] (v1) {};
    \node[vertex, fill=red!20,  label=above:{$v_2$}, right of=v1] (v2) {};
    \node[vertex, fill=green!20, label=below:{$v_3$}, below right of=v2] (v3) {};
    \node[vertex, fill=yellow!20, label=above:{$v_4$}, above right of=v3] (v4) {};
    
    % Draw directed edges with weights:
    % The 'node[weight]' places the label on the edge with white background to mask crossing lines
    \draw[-] (v1) to node[weight] {2.5} (v2);
    \draw[->] (v2) to node[weight] {1}   (v4);
    \draw[->] (v1) to node[weight] {4}   (v3);
    \draw[-] (v3) to node[weight] {3}   (v4);
    
    % An example of a bent edge with a weight:
    \draw[->, bend left=30] (v2) to node[weight] {5} (v3);
    
    \end{tikzpicture}
    \caption{Example of a directed weighted graph using TikZ. Some edges are drawn undirected.}
    \label{fig:exampletikzgraph}
\end{figure}



Also, describing algorithms is something you most likely will have to do. We suggest using the \lstinline|algorithmicx, algpseudocode| packages
for this. Below (\Cref{alg:example}) is an example.
\begin{algorithm}
    \caption{Sample Algorithm}\label{alg:example}
    \begin{algorithmic}
      \Procedure{Example}{$A$}
        \State $x \gets 0$
        \For{$i \gets 1$ to $n$}
          \If{$A[i] > x$}
            \State $x \gets A[i]$
          \EndIf
        \EndFor
        \State \Return $x$
      \EndProcedure
    \end{algorithmic}
    \end{algorithm}


\section{Other tips and tricks}
Now we will see some less common but nevertheless useful things one can do with \LaTeX{}. Perhaps the 
most important of these is the use of \emph{restatable} theorem environments. 
\begin{Rule}[Restatables] An academic paper typically has 
a ``Results'' section where the authors present the main theorems. Their proof is usually much later in the paper,
and it helps to have the statement of the theorem before the proof as well. However, if we simply copy paste the 
theorem, it would get a new number in the second occurence, as shown below.
\begin{theorem}
    This is a theorem with inconsistent numbering.
\end{theorem}
\begin{theorem}
    This is a theorem with inconsistent numbering.
\end{theorem}
The solution is in the \lstinline|thm-restate| package. Below we show how to use it, simply look at the code.
\begin{restatable}{theorem}{myRestatableTheoremCommand}
    \label{theorem:myRestatableTheorem}
    I can restate this.
    \end{restatable}
Now that we wrote the theorem for the first time, we can restate it and write its proof.
\myRestatableTheoremCommand*
\begin{proof}
    One can see.
\end{proof}
\end{Rule}
\begin{Rule}[Fractions and vertical spacing]\label{rule:vertical_spacing} It is very common to overuse the \lstinline|\frac{}{}| command. However, this creates some issues, especially 
when it is used in exponents of inline math. Notice the weird vertical spacing when we use $e^{\frac{n^2}{8}}$.
This happens because \LaTeX{} surrounds the expression with a ``box'' which pushes the line downwards so there 
is no overlap. A (partial) solution is to make this box shorter, by using for example $e^{n^2/8}$ instead. Or even $\Exp{n^2/8}$ 
to make the effect disappear completely. In general, try to limit the use of ``tall'' math inline, 
to maintain vertical spacing consistent. This also applies to the overuse of \lstinline|\left(...\right)|,
which might size the parentheses a bit too tall. This can happen implicitly if you overuse the starred version of the macros for Landau notation, et cetera. When using inline math, always choose the 
bracket sizes manually.
While we are at it, here is another tip on \lstinline|\frac{}{}|. 
There are also the commands \lstinline|\tfrac{}{}| and \lstinline|\dfrac{}{}|, which stand for text style (inline math)
and display style (display math). Using \lstinline|\dfrac{}{}| in inline math makes the fraction larger: $\dfrac{22}{7}$.
The functionality of \lstinline|\tfrac{}{}| is the reverse inside display math.
\end{Rule}

\begin{Rule} [Referring to equations]
    You can use the \lstinline|equation| environment to write down an important equation, numbering it so that you can refer 
    to it later.
    \begin{equation}\label{eq:einstein}
        E = mc^2 
    \end{equation}
    Who does not know~\cref{eq:einstein}? Excuse me, I meant~\Cref{eq:einstein}! Instead of using a number,
    one can also \emph{tag} an equation with a name.
    \begin{equation}\label{eq:force}
        F = ma \tag{force}
    \end{equation}
    \Cref{eq:force} is important too!
\end{Rule}

\begin{Rule} [Detexify]
Sometimes you just can't remember or find the latex code for a symbol (although both the Overleaf
and VSCode extension widgets work well for this). Thankfully, there is a tool online where you can draw
a math expression with your mouse and it gives you a list of candidate \LaTeX{} commands. \href{https://detexify.kirelabs.org/classify.html}{Here} it is.    
\end{Rule}

\begin{Rule} [Conditional \LaTeX{}]
    \LaTeX{} has a neat figure that allows you to set flags and include content conditionally on those flags.
    One of the most important use cases of this is when sending an article for anonymous review. 
    One can set an \lstinline|anon| flag based on which their name may be hidden from the final output.
    We give an example here:

    \newif\ifanon
    % Set the flag to false for now.
    \anonfalse

    % make a command for author name
    \newcommand{\authorname}{
        \ifanon ANON\else A.\ Uthor\fi
    }

    This manuscript was written by\authorname{}.
    \anontrue
    Now I cannot tell you that\authorname{} wrote the manuscript.

\end{Rule}