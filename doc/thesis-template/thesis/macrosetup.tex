%% Custom commands
%% ===============

%% Special characters for number sets, e.g. real or complex numbers.
\newcommand{\C}{\ensuremath{\mathbb{C}}}
\newcommand{\K}{\ensuremath{\mathbb{K}}}
\newcommand{\N}{\ensuremath{\mathbb{N}}}
\newcommand{\Q}{\ensuremath{\mathbb{Q}}}
\newcommand{\R}{\ensuremath{\mathbb{R}}}
\newcommand{\Z}{\ensuremath{\mathbb{Z}}}
\newcommand{\X}{\ensuremath{\mathbb{X}}}

%% Fixed/scaling delimiter examples (see mathtools documentation)
\DeclarePairedDelimiter\abs{\lvert}{\rvert}
\DeclarePairedDelimiter\norm{\lVert}{\rVert}

% Define generic paired delimiters:
\DeclarePairedDelimiter{\Paren}{(}{)}
\DeclarePairedDelimiter{\Bracket}{[}{]}
\DeclarePairedDelimiter{\Brace}{\{}{\}}


% Big-O
\NewDocumentCommand{\Oh}{som}{%
  \ensuremath{\mathcal{O}\IfBooleanTF{#1}{%
    \Paren*{#3}%          % *-version => auto-size
  }{%
    \IfNoValueTF{#2}{%
      \Paren{#3}%         % no optional => normal size
    }{%
      \Paren[#2]{#3}%     % optional => \big, \Big, etc.
    }%
  }%
}}

%Big Omega:
\NewDocumentCommand{\Om}{som}{%
\ensuremath{\Omega\IfBooleanTF{#1}{%
    \Paren*{#3}%
  }{%
    \IfNoValueTF{#2}{%
      \Paren{#3}%
    }{%
      \Paren[#2]{#3}%
    }%
  }%
}}

% Big Theta:
\NewDocumentCommand{\Th}{som}{%
\ensuremath{\Theta\IfBooleanTF{#1}{%
    \Paren*{#3}%
  }{%
    \IfNoValueTF{#2}{%
      \Paren{#3}%
    }{%
      \Paren[#2]{#3}%
    }%
  }%
}}

% little-o:
\NewDocumentCommand{\oh}{som}{%
\ensuremath{o\IfBooleanTF{#1}{%
    \Paren*{#3}%
  }{%
    \IfNoValueTF{#2}{%
      \Paren{#3}%
    }{%
      \Paren[#2]{#3}%
    }%
  }%
}}

% little-omega:
\NewDocumentCommand{\om}{som}{%
\ensuremath{\omega\IfBooleanTF{#1}{%
    \Paren*{#3}%
  }{%
    \IfNoValueTF{#2}{%
      \Paren{#3}%
    }{%
      \Paren[#2]{#3}%
    }%
  }%
}}

% Probability:  \Prob{Event} =>  P{Event},   \Prob*{...} => auto-sized { ... }
%    By default, we use curly braces for probabilities.
\NewDocumentCommand{\Prob}{som}{%
\ensuremath{\mathbb{P}% or \Pr, if you prefer
  \IfBooleanTF{#1}{%
    % *-version => auto-sized braces
    \Bracket*{#3}%
  }{%
    % no star
    \IfNoValueTF{#2}{%
      % no optional argument => normal size braces
      \Bracket{#3}%
    }{%
      % optional argument => \big, \Big, etc.
      \Bracket[#2]{#3}%
    }%
  }%
}}

% Expectation:  \E{X} => E[X],   \E*{X} => auto-sized [ X ]
\NewDocumentCommand{\E}{som}{%
\ensuremath{\mathbb{E}
  \IfBooleanTF{#1}{%
    \Bracket*{#3}%
  }{%
    \IfNoValueTF{#2}{%
      \Bracket{#3}%
    }{%
      \Bracket[#2]{#3}%
    }%
  }%
}}

\NewDocumentCommand{\Max}{s o m m}{%
\ensuremath{\max
  \IfBooleanTF{#1}{%
    \Brace*{#3,\, #4}%
  }{%
    \IfNoValueTF{#2}{%
      \Brace{#3,\, #4}%
    }{%
      \Brace[#2]{#3,\, #4}
    }%
  }%
}}

\NewDocumentCommand{\Min}{s o m m}{%
\ensuremath{\min
  \IfBooleanTF{#1}{%
    \Brace*{#3,\, #4}%
  }{%
    \IfNoValueTF{#2}{%
      \Brace{#3,\, #4}%
    }{%
      \Brace[#2]{#3,\, #4}
    }%
  }%
}}

% Exponential:  \Exp{x} => exp(x)
\NewDocumentCommand{\Exp}{som}{%
\ensuremath{\exp
  \IfBooleanTF{#1}{%
    \Paren*{#3}%
  }{%
    \IfNoValueTF{#2}{%
      \Paren{#3}%
    }{%
      \Paren[#2]{#3}%
    }%
  }%
}}

% You might want \Log, \Var, \Cov, etc. Similarly:
\NewDocumentCommand{\Log}{som}{%
\ensuremath{\log
  \IfBooleanTF{#1}{%
    \Paren*{#3}%
  }{%
    \IfNoValueTF{#2}{%
      \Paren{#3}%
    }{%
      \Paren[#2]{#3}%
    }%
  }%
}}

\NewDocumentCommand{\Var}{som}{%
\ensuremath{\mathrm{Var}
  \IfBooleanTF{#1}{%
    \Bracket*{#3}%
  }{%
    \IfNoValueTF{#2}{%
      \Bracket{#3}%
    }{%
      \Bracket[#2]{#3}%
    }%
  }%
}}


\NewDocumentCommand{\SetBuilder}{s o m m}{%
\ensuremath{\IfBooleanTF{#1}{%
    % Star version: auto-sized braces
    \Brace*{\,#3 : #4\,}%
  }{%
    % Non-star version
    \IfNoValueTF{#2}{%
      % No optional argument => normal-size braces
      \Brace{\,#3 : #4\,}%
    }{%
      % Optional argument => user-specified size, e.g. \big, \Big, \bigg
      \Brace[#2]{\,#3 : #4\,}%
    }%
  }%
}}



%% The xspace package is used to automatically handle spacing, regardless of whether the command is followed by punctuation or not.
\newcommand{\onemax}{\textsc{OneMax}\xspace}


%% \eps is quicker to type and the \varepsilon font is more common and nice.
\newcommand{\eps}{\ensuremath\varepsilon}

%% Also set the alternate phi as default, no real reason to use the old one, but one can.
\let\oldphi\phi
\renewcommand{\phi}{\ensuremath{\varphi}}
