%% (Master) Thesis template
% Template version used: v1.4
%
% Largely adapted from Adrian Nievergelt's template for the ADPS
% (lecture notes) project.


%% We use the memoir class because it offers a many easy to use features.
\documentclass[11pt,a4paper,oneside]{memoir}

%% Packages
%% ========

%% LaTeX Font encoding -- DO NOT CHANGE
\usepackage[OT1]{fontenc}

%% Babel provides support for languages.  'english' uses British
%% English hyphenation and text snippets like "Figure" and
%% "Theorem". Use the option 'ngerman' if your document is in German.
%% Use 'american' for American English.  Note that if you change this,
%% the next LaTeX run may show spurious errors.  Simply run it again.
%% If they persist, remove the .aux file and try again.
\usepackage[english]{babel}

%% Input encoding 'utf8'. In some cases you might need 'utf8x' for
%% extra symbols. Not all editors, especially on Windows, are UTF-8
%% capable, so you may want to use 'latin1' instead.
\usepackage[utf8]{inputenc}

%% This changes default fonts for both text and math mode to use Herman Zapfs
%% excellent Palatino font.  Do not change this.
\usepackage[sc]{mathpazo}


%% [REC] TikZ is used to draw figures
\usepackage{tikz}
% For positioning library (helps position nodes more easily), 
% and arrows.meta for more arrow tips and styles
\usetikzlibrary{positioning, arrows.meta}



%% The AMS-LaTeX extensions for mathematical typesetting.  Do not
%% remove.
\usepackage{amsmath,amssymb,amsfonts,mathrsfs}


%% NTheorem is a reimplementation of the AMS Theorem package. This
%% will allow us to typeset theorems like examples, proofs and
%% similar.  Do not remove.
%% NOTE: Must be loaded AFTER amsmath, or the \qed placement will
%% break
\usepackage[amsmath,thmmarks]{ntheorem}

%% LaTeX' own graphics handling
\usepackage{graphicx}

%% We unfortunately need this for the Rules chapter.  Remove it
%% afterwards; or at least NEVER use its underlining features.
\usepackage{soul}

%% This allows you to add .pdf files. It is used to add the
%% declaration of originality.
\usepackage{pdfpages}

%% Some more packages that you may want to use.  Have a look at the
%% file, and consult the package docs for each.
\input{extrapackages}

%% Our layout configuration.  DO NOT CHANGE.
\input{layoutsetup}

%% Theorem environments.  You will have to adapt this for a German
%% thesis.
\input{theoremsetup}

%% Helpful macros.
%% Custom commands
%% ===============

%% Special characters for number sets, e.g. real or complex numbers.
\newcommand{\C}{\ensuremath{\mathbb{C}}}
\newcommand{\K}{\ensuremath{\mathbb{K}}}
\newcommand{\N}{\ensuremath{\mathbb{N}}}
\newcommand{\Q}{\ensuremath{\mathbb{Q}}}
\newcommand{\R}{\ensuremath{\mathbb{R}}}
\newcommand{\Z}{\ensuremath{\mathbb{Z}}}
\newcommand{\X}{\ensuremath{\mathbb{X}}}

%% Fixed/scaling delimiter examples (see mathtools documentation)
\DeclarePairedDelimiter\abs{\lvert}{\rvert}
\DeclarePairedDelimiter\norm{\lVert}{\rVert}

% Define generic paired delimiters:
\DeclarePairedDelimiter{\Paren}{(}{)}
\DeclarePairedDelimiter{\Bracket}{[}{]}
\DeclarePairedDelimiter{\Brace}{\{}{\}}


% Big-O
\NewDocumentCommand{\Oh}{som}{%
  \ensuremath{\mathcal{O}\IfBooleanTF{#1}{%
    \Paren*{#3}%          % *-version => auto-size
  }{%
    \IfNoValueTF{#2}{%
      \Paren{#3}%         % no optional => normal size
    }{%
      \Paren[#2]{#3}%     % optional => \big, \Big, etc.
    }%
  }%
}}

%Big Omega:
\NewDocumentCommand{\Om}{som}{%
\ensuremath{\Omega\IfBooleanTF{#1}{%
    \Paren*{#3}%
  }{%
    \IfNoValueTF{#2}{%
      \Paren{#3}%
    }{%
      \Paren[#2]{#3}%
    }%
  }%
}}

% Big Theta:
\NewDocumentCommand{\Th}{som}{%
\ensuremath{\Theta\IfBooleanTF{#1}{%
    \Paren*{#3}%
  }{%
    \IfNoValueTF{#2}{%
      \Paren{#3}%
    }{%
      \Paren[#2]{#3}%
    }%
  }%
}}

% little-o:
\NewDocumentCommand{\oh}{som}{%
\ensuremath{o\IfBooleanTF{#1}{%
    \Paren*{#3}%
  }{%
    \IfNoValueTF{#2}{%
      \Paren{#3}%
    }{%
      \Paren[#2]{#3}%
    }%
  }%
}}

% little-omega:
\NewDocumentCommand{\om}{som}{%
\ensuremath{\omega\IfBooleanTF{#1}{%
    \Paren*{#3}%
  }{%
    \IfNoValueTF{#2}{%
      \Paren{#3}%
    }{%
      \Paren[#2]{#3}%
    }%
  }%
}}

% Probability:  \Prob{Event} =>  P{Event},   \Prob*{...} => auto-sized { ... }
%    By default, we use curly braces for probabilities.
\NewDocumentCommand{\Prob}{som}{%
\ensuremath{\mathbb{P}% or \Pr, if you prefer
  \IfBooleanTF{#1}{%
    % *-version => auto-sized braces
    \Bracket*{#3}%
  }{%
    % no star
    \IfNoValueTF{#2}{%
      % no optional argument => normal size braces
      \Bracket{#3}%
    }{%
      % optional argument => \big, \Big, etc.
      \Bracket[#2]{#3}%
    }%
  }%
}}

% Expectation:  \E{X} => E[X],   \E*{X} => auto-sized [ X ]
\NewDocumentCommand{\E}{som}{%
\ensuremath{\mathbb{E}
  \IfBooleanTF{#1}{%
    \Bracket*{#3}%
  }{%
    \IfNoValueTF{#2}{%
      \Bracket{#3}%
    }{%
      \Bracket[#2]{#3}%
    }%
  }%
}}

\NewDocumentCommand{\Max}{s o m m}{%
\ensuremath{\max
  \IfBooleanTF{#1}{%
    \Brace*{#3,\, #4}%
  }{%
    \IfNoValueTF{#2}{%
      \Brace{#3,\, #4}%
    }{%
      \Brace[#2]{#3,\, #4}
    }%
  }%
}}

\NewDocumentCommand{\Min}{s o m m}{%
\ensuremath{\min
  \IfBooleanTF{#1}{%
    \Brace*{#3,\, #4}%
  }{%
    \IfNoValueTF{#2}{%
      \Brace{#3,\, #4}%
    }{%
      \Brace[#2]{#3,\, #4}
    }%
  }%
}}

% Exponential:  \Exp{x} => exp(x)
\NewDocumentCommand{\Exp}{som}{%
\ensuremath{\exp
  \IfBooleanTF{#1}{%
    \Paren*{#3}%
  }{%
    \IfNoValueTF{#2}{%
      \Paren{#3}%
    }{%
      \Paren[#2]{#3}%
    }%
  }%
}}

% You might want \Log, \Var, \Cov, etc. Similarly:
\NewDocumentCommand{\Log}{som}{%
\ensuremath{\log
  \IfBooleanTF{#1}{%
    \Paren*{#3}%
  }{%
    \IfNoValueTF{#2}{%
      \Paren{#3}%
    }{%
      \Paren[#2]{#3}%
    }%
  }%
}}

\NewDocumentCommand{\Var}{som}{%
\ensuremath{\mathrm{Var}
  \IfBooleanTF{#1}{%
    \Bracket*{#3}%
  }{%
    \IfNoValueTF{#2}{%
      \Bracket{#3}%
    }{%
      \Bracket[#2]{#3}%
    }%
  }%
}}


\NewDocumentCommand{\SetBuilder}{s o m m}{%
\ensuremath{\IfBooleanTF{#1}{%
    % Star version: auto-sized braces
    \Brace*{\,#3 : #4\,}%
  }{%
    % Non-star version
    \IfNoValueTF{#2}{%
      % No optional argument => normal-size braces
      \Brace{\,#3 : #4\,}%
    }{%
      % Optional argument => user-specified size, e.g. \big, \Big, \bigg
      \Brace[#2]{\,#3 : #4\,}%
    }%
  }%
}}



%% The xspace package is used to automatically handle spacing, regardless of whether the command is followed by punctuation or not.
\newcommand{\onemax}{\textsc{OneMax}\xspace}


%% \eps is quicker to type and the \varepsilon font is more common and nice.
\newcommand{\eps}{\ensuremath\varepsilon}

%% Also set the alternate phi as default, no real reason to use the old one, but one can.
\let\oldphi\phi
\renewcommand{\phi}{\ensuremath{\varphi}}


%% Make document internal hyperlinks wherever possible. (TOC, references)
\usepackage[linkcolor=black,colorlinks=true,citecolor=black,filecolor=black]{hyperref}

% Load cleveref after hyperref
\usepackage{cleveref}

% We now name most things that are to be referred to.
% You can add similar commands for your own defined environments
% THEOREM
\crefname{theorem}{theorem}{theorems}
\Crefname{theorem}{Theorem}{Theorems}

% LEMMA
\crefname{lemma}{lemma}{lemmas}
\Crefname{lemma}{Lemma}{Lemmas}

% COROLLARY
\crefname{corollary}{corollary}{corollaries}
\Crefname{corollary}{Corollary}{Corollaries}

% PROPOSITION
\crefname{proposition}{proposition}{propositions}
\Crefname{proposition}{Proposition}{Propositions}

% CLAIM
\crefname{claim}{claim}{claims}
\Crefname{claim}{Claim}{Claims}

% DEFINITION
\crefname{definition}{definition}{definitions}
\Crefname{definition}{Definition}{Definitions}

% EXAMPLE
\crefname{example}{example}{examples}
\Crefname{example}{Example}{Examples}

% REMARK
\crefname{remark}{remark}{remarks}
\Crefname{remark}{Remark}{Remarks}

% RULE - This one is special
\crefname{Rule}{rule}{rules}
\Crefname{Rule}{Rule}{Rules}


%% [REC] This is for restating theorems.
\usepackage{thmtools}
\usepackage{thm-restate}      


%% [REC] packages for pseudocode
\usepackage{algorithm}        % A float wrapper for algorithms
\usepackage{algorithmicx}
\usepackage{algpseudocode}



%% IMPORTANT!!! Change the active line below to \studentwritingtrue
%% when you want to start writing to easily get rid of the initial chapters
\newif\ifstudentwriting
\studentwritingfalse
% \studentwritingtrue



%% Document information
%% ====================

\title{Title of Thesis}
\author{S. Tudent}
\thesistype{Master Thesis}
\advisors{Advisors: Prof.\ Dr.\ A. D. Visor, Dr.\ P. Ostdoc}
\department{Department of Computer Science}
\date{January 19, 2038}


\ifstudentwriting
\title{A Randomized Algorithm for Computing the Median}
\author{S. Tudent}
\thesistype{Master Thesis}
\advisors{Advisors: Prof.\ Dr.\ A. D. Visor, Dr.\ P. Ostdoc}
\department{Department of Computer Science}
\date{January 19, 2038}
\fi

\begin{document}

\frontmatter

%% Title page is autogenerated from document information above.  DO
%% NOT CHANGE.
\begin{titlingpage}
  \calccentering{\unitlength}
  \begin{adjustwidth*}{\unitlength-24pt}{-\unitlength-24pt}
    \maketitle
  \end{adjustwidth*}
\end{titlingpage}


\chapter*{Acknowledgments}
You can use this section to thank whoever you want to in the
context of your thesis. This can of course include family and friends,
colleagues, your advisor, and in general anyone that provide some 
form of support. You can also take the opportunity here to give credit
to people who may have enlightened you with discussions, and also certainly your co-authors 
of an academic paper related to the thesis, should there be one.
\pagebreak




\ifstudentwriting \begin{abstract}
  We present a randomized algorithm for selecting the median of an unsorted array in \Oh*{n} time with high probability. Unlike classical deterministic methods---notably the median-of-medians algorithm---our approach leverages random sampling to simplify the pivot-selection process and reduce overhead. Specifically, we extract a sublinear sample of the input, sort this sample, and employ its near-median elements as pivots. By bounding \Prob{\abs{X - \E{X}} \ge k} for appropriate choices of \(X\) and \(k\), we ensure that the size of the sub-array requiring a full sort remains sufficiently small.
  A rigorous analysis, based on Chebyshev’s inequality, demonstrates that the algorithm’s probability of “failure” (i.e., incorrectly bracketed pivots) decays quickly with the input size. In successful executions, the asymptotic cost of sorting the sample and the sub-array is dominated by a single pass through the input, yielding an overall running time of \Oh*{n}. While deterministic linear-time solutions to selection exist, our randomized method offers competitive performance in practice, owing to its simplicity and favorable hidden constants. We also discuss how this technique can be combined with deterministic strategies to eliminate worst-case risk.
\end{abstract} \else \begin{abstract}
    In this document we show you how to structure your paper and give some guidelines regarding
    your (Math and English) writing.
\end{abstract} \fi


%% TOC with the proper setup, do not change.
\cleartorecto
\tableofcontents
\mainmatter

\ifstudentwriting \else 
% Some commands used in this file
\newcommand{\package}{\emph}

\chapter{Introduction}

This is version \verb-v2.0- of the template.

We assume that you found this template on our institute's website, so
we do not repeat everything stated there (although some things 
are repeated for your convenience as far as \LaTeX{} code is concerned).
Consult the website again for pointers to further reading about \LaTeX{}.  This chapter only
gives a brief overview of the files you are looking at. Before we start, we would like to point 
out that once you read all the information, you can switch the flag \lstinline|studentwriting| (found in \lstinline|thesis.tex|) to true so that you start with a cleaner template
(basically starting from~\Cref{ch:paper_introduction}) in which you can fill in your thesis content.

\section{Features}
\label{sec:features}

The rest of this document shows off a few features of the template
files.  Look at the source code to see which macros we used!

The template is divided into \TeX{} files as follows:
\begin{enumerate}
\item \texttt{thesis.tex} is the main file.
\item \texttt{extrapackages.tex} holds extra package includes.
\item \texttt{layoutsetup.tex} defines the style used in this document.
\item \texttt{theoremsetup.tex} declares the theorem-like environments.
\item \texttt{macrosetup.tex} defines extra macros that you may find
  useful.
\item \texttt{introduction.tex} contains this text.
\item \texttt{sections.tex} is a quick demo of each sectioning level
  available (not included in the pdf, but you can consult it).
\item \texttt{refs.bib} is an example bibliography file.  You can use
  Bib\TeX{} to quote references.  For example, read
  \cite{bringhurst1996ets} if you can get a hold of it.
\end{enumerate}


\subsection{Extra package includes}

The file \texttt{extrapackages.tex} lists some packages that usually
come in handy.  Simply have a look at the source code.  We have
added the following comments based on our experiences:
\begin{description}
\item[REC] This package is recommended.
\item[OPT] This package is optional.  It usually solves a specific
  problem in a clever way.
\item[ADV] This package is for the advanced user, but solves a problem
  frequent enough that we mention it. Consult the package's
  documentation.
\end{description}

As a small example, here is a reference to the Section \emph{Features}
typeset with the recommended \package{cleverref} package. As the name implies,
this package cleverly identifies what the thing it is referring to is (Section, Theorem, Remark, etc.).
Note that we have not typed the word `Section' in the \LaTeX{} code.
See the code after \lstinline|\usepackage{cleveref}| in the \lstinline|thesis.tex| file for details.
\begin{quote}
  See~\Cref{sec:features}.
\end{quote}


\subsection{Layout setup}

This defines the overall look of the document -- for example, it
changes the chapter and section heading appearance.  We consider this
a `do not touch' area.  Take a look at the excellent 
\href{https://ctan.math.washington.edu/tex-archive/macros/latex/contrib/memoir/memman.pdf}{\emph{Memoir}
documentation} before changing it.


\subsection{Theorem setup}

This file defines a bunch of theorem-like environments.

\begin{lemma}\label{lem:example}
  An example lemma.
\end{lemma}

\begin{proof}
  Proof text goes here.
\end{proof}

Note that the q.e.d.\ symbol moves to the correct place automatically (because of \lstinline|ntheorem|)
if you end the proof with an \texttt{enumerate} or
\texttt{displaymath}.  You do not need to use \verb-\qedhere- as with
\package{amsthm}.

\begin{theorem}[Some Famous Guy]\label{theorem:example}
  An example theorem.
\end{theorem}

\begin{proof}
  This proof
  \begin{enumerate}
  \item ends in an enumerate.
  \end{enumerate}
\end{proof}

\begin{proposition}
  Note that all theorem-like environments are by default numbered on
  the same counter, i.e.\ we have~\Cref{lem:example} and then~\Cref{theorem:example}.
\end{proposition}

\begin{proof}
  This proof ends in a display like so:
  \begin{displaymath}
    f(x) = x^2.
  \end{displaymath}
\end{proof}


\subsection{Macro setup}

In the macro setup file we show how to define some basic macros,
and how to use a neat feature of the \package{mathtools} package:
\begin{displaymath}
  \abs{a}, \quad \abs*{\frac{a}{b}}, \quad \abs[\big]{\frac{a}{b}}.
\end{displaymath}

What is happening above is that whatever goes into the \lstinline|\abs| command gets
surrounded with vertical bars. If you add a \lstinline|*| to the command, it sizes the 
bars automatically to cover the whole expression. Or you can pass an argument to 
set the size yourself (i.e. \lstinline|\big|, \lstinline|\biggg|, \lstinline|\Bigg|, etc.), which is a good approach in many cases since the automatic version
is not always the best. We have also given macros for Landau notation.
\begin{align*}
  &\Oh{n^2}, \quad \Oh*{n^{\frac{1}{2}}}, \quad \Oh[\Bigg]{n^{\frac{1}{2}}}\\
  &\Om{n^3}, \quad \Om*{n^{e^{e^{4}}}}, \quad \Om[\big]{\log{n}}\\
  &\Th{n^4}, \quad \oh*{n^2 + n^{\frac{5n}{\log{n}}}}, \quad \om[\bigg]{\log^2{n}}
\end{align*}
When in doubt, use the starred version and then check to see if things look weird. In such 
cases, experiment with fixed sizes. After a while, you will be able to guess the correct size 
reliably. An example where it is better to use your own judgement (and which shows the differences
between automatic sizing with \lstinline|\left(...\right)| and manual with \lstinline|\biggl(...\biggr)| 
under the hood in our macros):
\[\left( \sum\limits_{i = 1}^{n} \right)^2.\]
Compare to the manually-sized:
\[\biggl( \sum\limits_{i = 1}^{n} \biggr)^2.\]
An exception to the rule of using the starred version is when writing inline math, where we don't want to use unnecessarily large vertical spacing, because that messes with the line spacing. For more details, see~\Cref{rule:vertical_spacing}.
Some more macros (it is also not hard to define your own if you need something extra):
\begin{align*}
  &\Prob*{v \text{ has degree 5}}, \quad \E*{X}, \quad \Max*{Y}{Z}\\
  &\Min*{X + Z}{Y + W}, \quad \Exp*{\log{n}}, \quad \Log*{x^{n^{d}}}, \quad \Var*{Z}
\end{align*}
We also define an example macro for when one needs to regularly refer to a word with a different font, like the \onemax problem. The use of the \lstinline|\xspace| command is important here, as it allows for correct spacing after the macro, regardless of whether it is followed by punctuation or not, both of which can happen when discussing \onemax.

Another thing we do in this file is to define \lstinline|\eps| to be $\varepsilon$ because this style is more common and nicer to look at (compare $\eps$ to $\epsilon$). 
We also redefine \lstinline|\phi| to be \lstinline|\varphi| ($\phi$). You can still use the ``old'' versions with \lstinline|\epsilon| and \lstinline|\oldphi| ($\oldphi$). There are other 
characters that have variations for example, $\rho$ versus $\varrho$. Now that we mention characters, a common mistake is to use $l$
instead of $\ell$ in math, try to avoid it.

We also define some more macros, which are explained in~\Cref{chapter:mathtips}.
\input{rules}
\input{typography}
\chapter{More Tips on Math/\LaTeX{}}\label{chapter:mathtips}

\section{The basics}
\begin{Rule} [Display math] You might have heard at some point that it is not a great idea to use \lstinline|$$...$$|. Indeed,
the preferred way is to use \lstinline|\[...\]| or equivalently \lstinline|\begin{displaymath}...\end{displaymath}|.
The problem with the former is some issues with spacing and also that the command
\lstinline|\fleqn| (which flushes math to the left) does not work.     
\end{Rule}
\begin{Rule}[Punctuation] Speaking about inline math and display math, keep in mind that one should have punctuation outside of inline 
math, but inside display math. That is, we say: The random variable $X$, which has expectation 5, is positive.
But we would say: the expectation of the random variable $X$, which satisfies
\[X > 0,\]
is 5.
\end{Rule}
\begin{Rule}[Roman font] For functions like $\log{n}, \cos{n}$, don't just write their name in math, i.e. $logn, cosn$. Instead,
use the macros when available (as here), or if they are not you can use \lstinline|\mathrm{function}|, which prints 
$\mathrm{function}$. This uses the roman font in math. Note that this is also what we have done for defining the \lstinline|\Var| macro.
\end{Rule}
\begin{Rule}[Dots] Do not use \lstinline|$(x_1, ..., x_n)$|. Instead, you should use \lstinline|$(x_1, \ldots, x_n)$|, i.e. $(x_1, \ldots, x_n)$ when you 
want lowered dots and \lstinline|$ A \times \cdots \times Z$| ($ A \times \cdots \times Z$) when you want centered dots.
\end{Rule}
\section{Some common math for reference/uniformity}
\begin{Rule}[Set building]A very common thing to do is define sets via some sort of condition, like so:
\[X = \SetBuilder*{x \in \R}{f(x) > 0}.\]
You can use our macro \lstinline|\Set| for this, which takes two arguments, the left and right-hand side
of the $:$ symbol. You can also replace this with \lstinline|\mid| ($\mid$), but make sure to be consistent.    
\end{Rule}

\begin{Rule}[Cases] Sometimes we want to define a piecewise function. This can be done with the \lstinline|\cases| environment.
\[
f(n) = \begin{cases}
            n, & \text{if $n < T/2$}\\
            T - n, & \text{otherwise}
 \end{cases}\]
\end{Rule}
\begin{Rule}[Matrices] Matrices are also super common, here are some examples:
\[\begin{pmatrix}
    1 & 2 & 3\\
    a & b & c
    \end{pmatrix}, \quad 
    \begin{bmatrix}
    1 & 2 & 3\\
    e & f & g
    \end{bmatrix}, \quad
    \begin{Bmatrix}
    1 & 2 & 3\\
    h & i & j
    \end{Bmatrix}\]
One thing that you might not know is that you can have small matrices for inline math, like so: $\big(\begin{smallmatrix}
  k & l\\
  m & n
\end{smallmatrix}\big)$.
\end{Rule}
\begin{Rule}[Binomials] Finally, it is highly likely that you use binomial coefficients somewhere in your thesis, like so:
\[\binom{n}{k}.\]
\end{Rule}
\section{Figures and Algorithms}
Almost always it is a good idea to use figures to illustrate your arguments and help the 
reader understand them more deeply. For this, you can use TikZ, documented \href{https://tikz.dev/}{here}. Below we give 
an example figure (\Cref{fig:exampletikzgraph}) with some elements that might be useful. For other examples, look at the documentation (it is actually good 
and not just incomprehensible walls of text, i.e.\ it also has examples you can modify to your needs easily).


\begin{figure}[ht]
    \centering
    \begin{tikzpicture}[
        % Define common TikZ styles
        >=Stealth,              % Use Stealth arrow tips
        auto,                   % Position edge labels automatically
        node distance=2cm,      % Default distance between nodes
        thick,                  % Default line thickness
        every node/.style={font=\small},  % Node font size
        vertex/.style={
            circle, 
            draw, 
            minimum size=1.2em,
            inner sep=2pt
        },
        weight/.style={
            fill=white,
            inner sep=2pt
        }
    ]
    
    % Define vertices:
    % The 'label' key can be used to add labels outside the vertex;
    % an alternative is to put the label text directly in the node content (e.g., {A})
    \node[vertex, fill=blue!20, label=above:{$v_1$}] (v1) {};
    \node[vertex, fill=red!20,  label=above:{$v_2$}, right of=v1] (v2) {};
    \node[vertex, fill=green!20, label=below:{$v_3$}, below right of=v2] (v3) {};
    \node[vertex, fill=yellow!20, label=above:{$v_4$}, above right of=v3] (v4) {};
    
    % Draw directed edges with weights:
    % The 'node[weight]' places the label on the edge with white background to mask crossing lines
    \draw[-] (v1) to node[weight] {2.5} (v2);
    \draw[->] (v2) to node[weight] {1}   (v4);
    \draw[->] (v1) to node[weight] {4}   (v3);
    \draw[-] (v3) to node[weight] {3}   (v4);
    
    % An example of a bent edge with a weight:
    \draw[->, bend left=30] (v2) to node[weight] {5} (v3);
    
    \end{tikzpicture}
    \caption{Example of a directed weighted graph using TikZ. Some edges are drawn undirected.}
    \label{fig:exampletikzgraph}
\end{figure}



Also, describing algorithms is something you most likely will have to do. We suggest using the \lstinline|algorithmicx, algpseudocode| packages
for this. Below (\Cref{alg:example}) is an example.
\begin{algorithm}
    \caption{Sample Algorithm}\label{alg:example}
    \begin{algorithmic}
      \Procedure{Example}{$A$}
        \State $x \gets 0$
        \For{$i \gets 1$ to $n$}
          \If{$A[i] > x$}
            \State $x \gets A[i]$
          \EndIf
        \EndFor
        \State \Return $x$
      \EndProcedure
    \end{algorithmic}
    \end{algorithm}


\section{Other tips and tricks}
Now we will see some less common but nevertheless useful things one can do with \LaTeX{}. Perhaps the 
most important of these is the use of \emph{restatable} theorem environments. 
\begin{Rule}[Restatables] An academic paper typically has 
a ``Results'' section where the authors present the main theorems. Their proof is usually much later in the paper,
and it helps to have the statement of the theorem before the proof as well. However, if we simply copy paste the 
theorem, it would get a new number in the second occurence, as shown below.
\begin{theorem}
    This is a theorem with inconsistent numbering.
\end{theorem}
\begin{theorem}
    This is a theorem with inconsistent numbering.
\end{theorem}
The solution is in the \lstinline|thm-restate| package. Below we show how to use it, simply look at the code.
\begin{restatable}{theorem}{myRestatableTheoremCommand}
    \label{theorem:myRestatableTheorem}
    I can restate this.
    \end{restatable}
Now that we wrote the theorem for the first time, we can restate it and write its proof.
\myRestatableTheoremCommand*
\begin{proof}
    One can see.
\end{proof}
\end{Rule}
\begin{Rule}[Fractions and vertical spacing]\label{rule:vertical_spacing} It is very common to overuse the \lstinline|\frac{}{}| command. However, this creates some issues, especially 
when it is used in exponents of inline math. Notice the weird vertical spacing when we use $e^{\frac{n^2}{8}}$.
This happens because \LaTeX{} surrounds the expression with a ``box'' which pushes the line downwards so there 
is no overlap. A (partial) solution is to make this box shorter, by using for example $e^{n^2/8}$ instead. Or even $\Exp{n^2/8}$ 
to make the effect disappear completely. In general, try to limit the use of ``tall'' math inline, 
to maintain vertical spacing consistent. This also applies to the overuse of \lstinline|\left(...\right)|,
which might size the parentheses a bit too tall. This can happen implicitly if you overuse the starred version of the macros for Landau notation, et cetera. When using inline math, always choose the 
bracket sizes manually.
While we are at it, here is another tip on \lstinline|\frac{}{}|. 
There are also the commands \lstinline|\tfrac{}{}| and \lstinline|\dfrac{}{}|, which stand for text style (inline math)
and display style (display math). Using \lstinline|\dfrac{}{}| in inline math makes the fraction larger: $\dfrac{22}{7}$.
The functionality of \lstinline|\tfrac{}{}| is the reverse inside display math.
\end{Rule}

\begin{Rule} [Referring to equations]
    You can use the \lstinline|equation| environment to write down an important equation, numbering it so that you can refer 
    to it later.
    \begin{equation}\label{eq:einstein}
        E = mc^2 
    \end{equation}
    Who does not know~\cref{eq:einstein}? Excuse me, I meant~\Cref{eq:einstein}! Instead of using a number,
    one can also \emph{tag} an equation with a name.
    \begin{equation}\label{eq:force}
        F = ma \tag{force}
    \end{equation}
    \Cref{eq:force} is important too!
\end{Rule}

\begin{Rule} [Detexify]
Sometimes you just can't remember or find the latex code for a symbol (although both the Overleaf
and VSCode extension widgets work well for this). Thankfully, there is a tool online where you can draw
a math expression with your mouse and it gives you a list of candidate \LaTeX{} commands. \href{https://detexify.kirelabs.org/classify.html}{Here} it is.    
\end{Rule}

\begin{Rule} [Conditional \LaTeX{}]
    \LaTeX{} has a neat figure that allows you to set flags and include content conditionally on those flags.
    One of the most important use cases of this is when sending an article for anonymous review. 
    One can set an \lstinline|anon| flag based on which their name may be hidden from the final output.
    We give an example here:

    \newif\ifanon
    % Set the flag to false for now.
    \anonfalse

    % make a command for author name
    \newcommand{\authorname}{
        \ifanon ANON\else A.\ Uthor\fi
    }

    This manuscript was written by\authorname{}.
    \anontrue
    Now I cannot tell you that\authorname{} wrote the manuscript.

\end{Rule}
\chapter{How a Typical Academic Paper is Structured}\label{ch:structure}

Before delving into a concrete example, it is useful to understand the common structure of an academic paper—especially in the fields of computer science and mathematics. While specific requirements can vary across disciplines, journals, or conferences, most technical papers and theses follow a broadly similar organizational pattern. This chapter serves as an overview of that structure, highlighting the purpose and typical contents of each major section.

\section{Preamble}
\begin{itemize}
    \item \textbf{Title:} A concise statement reflecting the main topic or contribution. It should be both informative and engaging, often hinting at the central findings or approach.
    \item \textbf{Acknowledgments:} A short section/chapter to recognize the individuals, organizations, or funding agencies that supported the research. Some publications (conference or journal styles) place this section at the end, while others place it in a footnote on the first page.    
    \item \textbf{Abstract:} A brief (usually 150--300 words) summary of the paper. The abstract succinctly states the \emph{problem}, the \emph{approach} used to tackle the problem, and the \emph{main results/contributions}. Readers often decide whether to read the entire paper based on the clarity and relevance of the abstract.
\end{itemize}

\section{Introduction}
The introduction typically provides:
\begin{itemize}
    \item \textbf{Context and Motivation:} Explains why the topic is important, giving background or real-world relevance. The existence of many previous works studying related questions is also sometimes used as further motivation.
    \item \textbf{Problem Statement:} Defines the core question or problem addressed by the paper.
    \item \textbf{Prior Work or Background:} Summarizes existing literature and situates the problem within a broader research context. (Some papers have a separate “Related Work” section for more extensive discussion of prior research, if necessary.)
    \item \textbf{Contributions and Overview:} Concludes by stating what new insight, technique, or theoretical result the paper offers (listing the theorems, possibly restatable), and outlines the paper’s structure to guide readers through the subsequent sections.
\end{itemize}

\section{Notation and Preliminaries}
\begin{itemize}
    \item \textbf{Notation and Definitions:} Introduces the notation, terminology, or foundational concepts essential for understanding the paper’s main results.
    \item \textbf{Theoretical Tools:} Reviews any theorems, lemmas, or existing results (from probability, linear algebra, combinatorics, etc.) that the paper will rely on.
    \item \textbf{Contextual Details:} In some fields, this section might include domain-specific information (e.g., details on data models, assumptions about hardware, or historical context). Here you might also more formally define the problem/question studied, in case you opted for a more intuitive explanation in the introduction (which is a good approach).
\end{itemize}

\section{Main Content (Approach/Analysis)}
This is usually the largest and most detailed section. Depending on the nature of the work, it might be subdivided in various ways:

\subsection{Algorithm or Method Description}
\begin{itemize}
    \item \textbf{High-Level Idea:} An intuitive, conceptual overview of the proposed algorithm or approach.
    \item \textbf{Pseudocode or Formal Specification:} A step-by-step procedure or equations describing how the algorithm operates or how the proof of the result is structured.
\end{itemize}

\subsection{Theoretical Analysis}
\begin{itemize}
    \item \textbf{Correctness/Runtime/Theorem Proofs:} Here the formal proofs are given in all detail. Typically there is still a storyline to be followed in text to not lose the reader in the math, but each proof environment is now fully formal and self-standing. You can think of the explanations along the way like comments in code and the theorems/proofs as the actual code, which has to follow strict rules.
\end{itemize}

\subsection{Experimental Results (If Applicable)}
For more empirically focused papers:
\begin{itemize}
    \item \textbf{Dataset Description:} Overview of the data used for experiments, how it was collected, and why it is suitable.
    \item \textbf{Metrics:} Explanation of how performance or accuracy is measured.
    \item \textbf{Comparison with Baselines:} Charts, tables, or graphs illustrating how the proposed method stacks up against existing methods.
\end{itemize}

\section{Conclusion and Future Work}
The conclusion typically:
\begin{itemize}
    \item \textbf{Summarizes Main Findings:} Restates the key contributions and what has been learned.
    \item \textbf{Discusses Limitations:} Briefly considers conditions or scenarios where the proposed method might not perform as well.
    \item \textbf{Suggests Future Directions:} Outlines open questions or new problems raised by the work, offering pathways for continued research.
\end{itemize}


\section{References}
\begin{itemize}
    \item \textbf{Citations:} Every work mentioned in the paper (including background literature, data sources, and tools) should be cited in a consistently formatted bibliography.
    \item \textbf{Formatting Style:} Whether it’s APA, IEEE, ACM, Chicago, or another style, references should follow a recognized standard and be clear to the reader. For the thesis (and most CS conferences), you can use the citing style found in this template.
\end{itemize}

\section{Appendices (If Needed)}
\begin{itemize}
    \item \textbf{Supplementary Material:} Large/overly technical proofs, extended data tables, or technical details that disrupt the narrative flow can be placed in appendices to keep the main text concise.
    \item \textbf{Extended Examples:} Additional or more detailed examples that illustrate nuances of the proposed approach.
\end{itemize}

\section*{In This Template}
In the chapters that follow, we demonstrate how these broad guidelines translate into a concrete academic document. You will see:
\begin{itemize}
    \item An \textbf{Abstract} that briefly states the problem, the approach, and the main findings.
    \item An \textbf{Introduction} providing motivation and an overview of prior work.
    \item A chapter on \textbf{Notation and Preliminaries} that sets up key definitions and outlines the main probabilistic tool—Chebyshev’s inequality—used in the analysis.
    \item A detailed \textbf{Algorithm and Analysis} section illustrating the design of a randomized median-finding procedure and its theoretical guarantees.
    \item A concise \textbf{Conclusion} summarizing contributions and suggesting future research directions.
\end{itemize}

This organization—while slightly customized for our particular focus on randomized selection algorithms—follows the same general principles outlined above. Now that we have an overview of the typical structure, we can move on to our specific example of how these elements come together in a coherent academic work. Note that by default the abstract and title of the ``paper'' are not shown.
To see them (and also start filling in your thesis content), set the \lstinline|\ifstudentwriting| flag to true in \lstinline|thesis.tex|.

 \fi


\newpage

\chapter{Introduction}\label{ch:paper_introduction}

The median of a dataset is a fundamental measure of central tendency that arises in numerous computational and statistical contexts. In an array of \(n\) elements, indexed from 1 to \(n\), the median can be intuitively understood as the “middle” element in the sorted order if \(n\) is odd, or the average of the two middle elements if \(n\) is even. While this definition naturally suggests sorting the entire array, doing so in \Oh*{n \Log*{n}} time is often unnecessary when the goal is to identify only the median.

\section*{Motivation and Previous Work}

Several well-known algorithms exist for selecting the median or, more generally, the \(k\)-th smallest element of an array. A straightforward method is to sort the array and then pick the median in constant time. However, this approach has a runtime of \Oh*{n \Log*{n}}, which can be excessive for large \(n\). More advanced methods, such as the \emph{median-of-medians} algorithm introduced by Blum, Floyd, Pratt, Rivest, and Tarjan~\cite{blum1973time}, achieve a worst-case runtime of \Oh*{n} using a deterministic divide-and-conquer strategy. While optimal in the worst case, the median-of-medians algorithm involves significant overhead, making it less practical in many scenarios. Alternatively, randomized algorithms like Quickselect often achieve linear-time performance on average but can degrade to \Oh*{n^2} in the worst case without additional precautions.

\section*{Our Contribution}

In this thesis, we present a \emph{randomized} algorithm that computes the median in \Oh*{n} time \emph{with high probability}, thereby combining practical efficiency with strong theoretical guarantees. The key idea is to leverage a carefully chosen random sample of the input array:
\begin{enumerate}
    \item We extract a subset of elements (the “sample”) whose size grows sublinearly with \(n\).
    \item Using this sample, we identify pivot elements that, with high probability, closely approximate the true median's position.
    \item We then reduce the problem to sorting only a small subset of the original array, thereby achieving near-linear time complexity.
\end{enumerate}

Central to our analysis is the use of \textbf{Chebyshev’s inequality}. While often introduced in purely statistical contexts, Chebyshev’s inequality provides a useful probabilistic bound for ensuring that the sampled elements approximate the median accurately. In essence, it tells us that \Prob{\abs{X - \E{X}} \ge k} can be kept small if \Var{X} is finite, thereby giving us high confidence that our pivot-based “zoom-in” on the median is correct.

More concretely, we prove the following theorem with our algorithm.

\begin{restatable}{theorem}{mainTheoremCommand}
    \label{theorem:mainTheorem}
    There exists a randomized algorithm to compute the median that runs in time $\Oh*{n}$ and succeeds with probability $1 - \Oh*{n^{-1/4}}$.
\end{restatable}

\section*{Thesis Outline}

This thesis is structured as follows:
\begin{itemize}
    \item \textbf{\Cref{ch:notation} (Notation and Preliminaries):} We define our notational conventions and offer a brief refresher on Chebyshev’s inequality, setting the stage for the randomized analysis to come.
    \item \textbf{\Cref{ch:analysis} (Algorithm and Analysis):} We detail the randomized median-finding algorithm, present its pseudocode, and provide a rigorous proof of correctness. We also analyze the runtime and show why it remains \Oh*{n} with high probability.
    \item \textbf{\Cref{ch:conclusion} (Conclusion):} We summarize our findings, discuss the algorithm’s theoretical and practical implications, and highlight possible directions for future research.
\end{itemize}


\chapter{Notation and Preliminaries}\label{ch:notation}

\section*{Definitions and Notation}

\subsection*{Arrays and Median}
Throughout this thesis, we denote the input array of size \(n\) by
\[
a = (a[1], a[2], \dots, a[n]).
\]
We may also use $a[1 \dots n]$ to denote an array $a$ of length $n$.
Our goal is to find the \emph{median} of \(a\). Formally, we define:
\begin{itemize}
    \item If \(n\) is odd, the median is the element in the \(\lceil n/2 \rceil\)-th position of the array when sorted in non-decreasing order.
    \item If \(n\) is even, the median is the element in the \(\lceil n/2 \rceil\)-th position of the sorted array.
\end{itemize}

\subsection*{Random Variables and Expectation}
In the analysis of our randomized median-finding algorithm, we shall frequently refer to \emph{random variables}. A random variable \(X\) is a function from the sample space \(\Omega\) (representing all possible outcomes of a probabilistic experiment) to the real numbers \R.

\paragraph{Expectation.}
The \emph{expected value}, or \emph{mean}, of \(X\) is defined by
\[
\E{X} = \sum_{\omega \in \Omega} X(\omega)\,\Prob{\omega},
\]
in the discrete case. If \(X\) has a probability density function \(f_X\) (in the continuous case), then
\[
\E{X} = \int_{-\infty}^{\infty} x \, f_X(x) \, dx.
\]
Intuitively, \(\E{X}\) indicates the “average” or “central” value of the random variable over many independent trials.

\subsection*{Variance and Standard Deviation}
To measure how much a random variable deviates from its mean, we use its \emph{variance}, denoted \(\Var{X}\). This is defined as:
\[
\Var{X} = \E{\bigl(X - \E{X}\bigr)^2}.
\]
The \emph{standard deviation} of \(X\), denoted \(\sigma_X\), is the square root of the variance:
\[
\sigma_X = \sqrt{\Var{X}}.
\]
A smaller variance (or standard deviation) means the random variable’s values tend to lie closer to its mean, whereas a larger variance indicates a broader spread of values.

\subsection*{Asymptotic Notation}
In describing the time complexity of algorithms, we use \Oh*{\dots} notation in a standard way:
\[
f(n) = \Oh*{g(n)} 
\quad \Longleftrightarrow \quad
\lim_{n \to \infty} \frac{f(n)}{g(n)} < \infty.
\]
We may also use \Th{\ldots}, \Om{\ldots}, and other asymptotic notations where necessary.

\section*{Chebyshev’s Inequality}

\emph{Chebyshev’s inequality} is a cornerstone result in probability theory, providing a quantitative bound on how a random variable deviates from its expected value. Formally, let \(X\) be any random variable with finite mean \(\E{X}\) and finite variance \(\Var{X}\). Then, for any \(k > 0\),
\[
\Prob{\abs{X - \E{X}} \ge k} 
\;\le\; 
\frac{\Var{X}}{k^2}.
\]

\subsection*{Intuition and Relation to Markov’s Inequality}
To build intuition, recall the simpler \emph{Markov’s inequality}, which states that for any nonnegative random variable \(Y\) and \(t>0\),
\[
\Prob{Y \ge t} \le \frac{\E{Y}}{t}.
\]
Chebyshev’s inequality is often viewed as an application of Markov’s inequality to the random variable \((X - \E{X})^2\). Indeed,
\[
\Prob{(X - \E{X})^2 \ge k^2} 
\le 
\frac{\E{\bigl(X - \E{X}\bigr)^2}}{k^2}
= 
\frac{\Var{X}}{k^2},
\]
and from
\(\{\,|X - \E{X}| \ge k\,\} \,\equiv\, \{\,(X - \E{X})^2 \ge k^2\,\},\)
we obtain Chebyshev’s inequality.

The key takeaway is that Chebyshev’s inequality captures a broad class of distributions. It requires no assumption of normality or other specific distributional shapes, only finite variance. In exchange for this generality, Chebyshev’s bound can be somewhat loose compared to more specialized inequalities (e.g., Hoeffding’s, Chernoff’s, or Bernstein’s). Nonetheless, it remains a powerful and classic tool for bounding “tail events,” i.e.\ events in which \(X\) significantly deviates from its mean.

\subsection*{Why We Use Chebyshev’s Inequality Here}
In our randomized median-finding algorithm, we draw random samples from the input array and use them to approximate the true median. The “rank” of our chosen pivots in the sorted order of the entire array are themselves random variables. We want to show that, with high probability, these rank containt the \(\lceil n/2\rceil\)-th position.

By applying Chebyshev’s inequality, we obtain a bound on \Prob{\abs{X - \E{X}} \ge k} for a well-chosen \(k\), where \(X\) is the rank of our pivot.
In other words, Chebyshev’s inequality formally establishes the reliability of our sampling strategy by demonstrating that, except with small probability, our pivots partition the array around the actual median. This allows us to restrict further computation to a subset of the array whose size is guaranteed to be sufficiently small, thereby ensuring linear running time.

\subsection*{Extensions and Alternatives}
While Chebyshev’s inequality is sufficient for our purposes, more refined analyses might employ other concentration bounds such as Hoeffding’s or Chernoff’s inequalities. These stronger bounds can yield tighter probabilities when the underlying sample distribution has additional regularity properties (e.g., bounded random variables or independent Bernoulli trials). However, Chebyshev’s inequality has the advantage of simplicity and broad applicability, making it an ideal choice for illustrating the core ideas in a clear manner.


\chapter{Algorithm and Analysis}
\label{ch:analysis}



In this chapter, we present our main randomized median-finding algorithm in full detail and analyze its performance. We first give a high-level description and provide the pseudocode in Algorithm~\ref{alg:randmedian}. We then prove two central theorems: one that establishes the algorithm’s probability of success (i.e.\ returning the correct median) and another that bounds its running time by \Oh*{n}. Throughout this chapter, we use the notation introduced in Chapter~\ref{ch:notation}.
\section*{Algorithm Description}

The guiding idea behind the algorithm is to select a “small” random sample from the input array, sort this sample, and use carefully chosen elements from this sample as \emph{pivots} to bracket the true median. Specifically:
\begin{itemize}
    \item We draw \( m = \lceil n^{3/4} \rceil \) elements uniformly at random from the input array \( a[1 \dots n] \) and sort these sampled elements into an array \( b[1 \dots m] \).
    \item We then pick two elements from \( b \), denoted \( p_\ell \) and \( p_r \), near the middle of \( b \) but offset by \(\pm \sqrt{n}\). Intuitively, these two pivots should “sandwich” the true median of the entire array with high probability.
    \item We gather all elements of the original array \( a \) that lie between these two pivots (inclusive) into a temporary array \(T\). We verify that the median must lie in this sub-array, and that \(T\) is not excessively large.
    \item If those checks are satisfied, we sort \(T\) and extract the median; otherwise, the algorithm reports a \emph{failure}.
\end{itemize}
\begin{algorithm}
    \caption{RandMedian: Randomized Median Algorithm}
    \label{alg:randmedian}
    \begin{algorithmic}[1]
    \Require Unsorted array \( a[1 \dots n] \) of size \( n \).
    \Ensure Median of \( a \) or \emph{failure}.
    
    \State \textbf{Initialize} an empty array \(b\).
    \State \textbf{For each} element \(x \in a[1 \dots n]\):
    \Statex \quad \textbf{Include} \(x\) in \(b\) \textbf{with probability} \(n^{-1/4}\), independently of other elements.
    \State \( m \gets |b| \) 
        \Comment{(Random) size of the sample; note that \(\E{m} = n^{3/4}\).}

    \If{\( m > n^{3/4} + \sqrt{n} \)}
        \State \Return \emph{failure}
    \EndIf

    \State \textbf{Sort} \( b \) in \(\Oh*{m \log m}\) time
    
    \State \textbf{Compute pivots}:
    \[
      p_\ell \gets b\Bigl[\Bigl\lfloor \tfrac{m}{2} - \sqrt{n} \Bigr\rfloor\Bigr], 
      \quad
      p_r \gets b\Bigl[\Bigl\lceil \tfrac{m}{2} + \sqrt{n} \Bigr\rceil\Bigr]
    \]
    
    \State Define the subset \( T \gets \{\, x \in a : p_\ell \le x \le p_r \}\)
    
    \State Compute:
    \begin{align*}
        s &\gets \#\{\, x \in a : x < p_\ell \},\\
        t &\gets \#\{\, x \in a : p_\ell \le x \le p_r \}.
    \end{align*}
    
    \If{\( s < \lceil n/2 \rceil \)\textbf{ and }\( s + t \ge \lceil n/2 \rceil \)\textbf{ and }\( t < 4n^{3/4}\)} \label{alg:randmedian:check}
        \State \textbf{Sort} \( T \) in \(\Oh*{|T|\log |T|}\) time
        \State \(\text{Median of } a \;\gets\; T[\lceil n/2 \rceil - s]\)
        \State \Return \(\text{Median of } a\)
    \Else
        \State \Return \emph{failure}
    \EndIf
    \end{algorithmic}
    \end{algorithm}
    

The rest of this chapter gives a proof of two properties of this algorithm:
\begin{enumerate}
    \item With high probability, Algorithm~\ref{alg:randmedian} does not fail and returns the correct median.
    \item Conditioned on success, the overall running time is \Oh*{n}.
\end{enumerate}

\section*{Probability of Success}





\begin{figure}[htbp]
    \centering
    \begin{tikzpicture}[>=stealth, scale=1]
  
        % 1) Draw a big rectangle for array 'a'
        \draw[thick] (0,0) rectangle (8,1);
        \node[above] at (4,1) {\Large Array $a$};
  
        % 2) Draw a smaller rectangle for array 'b' (below 'a')
        \draw[thick] (1,-2) rectangle (7,-3);
        \node[above] at (4,-2) {\Large Array $b$};
  
        % 3) Mark p_l and p_r in 'b'
        \draw[thick] (2.5,-2) -- (2.5,-3); 
        \draw[thick] (5.5,-2) -- (5.5,-3); 
        \node[below] at (2.5,-3) {$p_\ell$};
        \node[below] at (5.5,-3) {$p_r$};
  
        % 4) Arrows from pivots in 'b' up to partitions in 'a'
        \draw[dashed,->] (2.5,-2) -- (2,0);
        \draw[dashed,->] (5.5,-2) -- (6,0);
  
        % Vertical lines in 'a' to show partition
        \draw[thick] (2,0) -- (2,1); 
        \draw[thick] (6,0) -- (6,1); 
  
        % 5) Label the segments in 'a'
        \node at (1,0.5) {$s$ elements};
        \node at (4,0.5) {subset of size $t$};
        \node at (7,0.5) {rest};
  
    \end{tikzpicture}
    \caption{Illustration of the arrays and the pivots, showing array $a$, sampled array $b$, and the pivots $p_\ell, p_r$. Note that this visualization shows the sorted version of the arrays.}
    \label{fig:randmedian}
  \end{figure}
  






We begin by showing that, except with small probability, the two pivots \(p_\ell\) and \(p_r\) are chosen so that the sub-array \(T\) contains the true median of \(a\) and is not excessively large. Formally, we have:

\begin{theorem}[Correctness with High Probability]
    \label{thm:correctness}
    The probability that~\Cref{alg:randmedian} returns the correct median is at least \(1 - \Oh*{n^{-1/4}}\). In other words, the algorithm succeeds \emph{with high probability}.
    \end{theorem}
    
    \begin{proof}
    For simplicity, we assume throughout this proof that \(n\) is even and that \(n^{3/4}\) is an integer; this avoids cumbersome floor and ceiling functions. The general case follows by analogous, but slightly more tedious, arguments.
    
    Recall that \(a[\,1\dots n\,]\) is our input array of \(n\) distinct elements. Let \(b[\,1\dots m\,]\) be the sample of size \(m\) drawn by sampling each element with probability $n^{-1/4}$ from \(a\). The setup is also visualized in~\Cref{fig:randmedian}. We then sort \(b\), pick the pivots\footnote{Technically, if $m$ is too small, the following lines are undefined (out of bounds). But this is avoided whp, as for the other events later in the proof.}
    \[
    p_\ell \;=\; b\Bigl[\,\tfrac{m}{2} - \sqrt{n}\Bigr],
    \quad
    p_r \;=\; b\Bigl[\,\tfrac{m}{2} + \sqrt{n}\Bigr],
    \]
    and collect into \(T\) all elements \(x \in a\) with \(p_\ell \le x \le p_r\). Let
    \[
    s \;=\; \#\SetBuilder{x\in a}{x < p_\ell}
    \quad\text{and}\quad
    t \;=\; \#\SetBuilder{x\in a}{p_\ell \le x \le p_r}.
    \]
    \paragraph{Key sets and random variables.}
For any integer \(k \in \{0,\dots,n\}\), define
\[
   S_k \;=\; \{\text{the }k \text{ smallest elements of } a\}.
\]
We then define the random variable
\[
   X_k \;=\; \#\{\, x \in b : x \in S_k \}.
\]
Since each element of \(a\) is independently included in \(b\) with probability \(n^{-1/4}\),
the number of elements of \(S_k\) that appear in \(b\) follows
\[
   X_k \;\sim\; \mathrm{Bin}\bigl(k,\;n^{-1/4}\bigr).
\]
Hence,
\[
   \E{X_k} \;=\; k\,n^{-1/4},
   \quad
   \Var{X_k} \;=\; k\,n^{-1/4}\bigl(1 - n^{-1/4}\bigr)
   \;\le\;
   k\,n^{-1/4}.
\]

    
    \paragraph{Application of Chebyshev’s Inequality.}
    Since \(X_k \sim \mathrm{Bin}(k,\,n^{-1/4})\) with 
\(\Var{X_k} \le k\,n^{-1/4},\)
Chebyshev’s inequality gives
\begin{equation}
    \Prob{\bigl|X_k - k\,n^{-1/4}\bigr|\;\ge\;\sqrt{n}}
    \;\;\le\;\;
    \frac{\Var{X_k}}{\,n\,}
    \;\;\le\;\;
    \frac{k\,n^{-1/4}}{n}
    \;=\;
    \frac{k}{n^{5/4}}. \label{eq:fail}   
\end{equation}
    \paragraph{Defining the “bad” events.}
    We show that the only way the algorithm can fail (i.e.\ not return the correct median) is if one of the following four events occurs. Denote
    \begin{align*}
    A \;&:=\; \{\, s \;\ge\; n/2 \}\,,
    \\
    B \;&:=\; \{\, s + t < n/2 \}\,,
    \\
    C \;&:=\; \{\, s < \tfrac{n}{2} - 2\,n^{3/4} \}\,,
    \\
    D \;&:=\; \{\, s + t \;\ge\; \tfrac{n}{2} + 2\,n^{3/4} \}\,.
    \\
    E \;&:=\; \{\, m > n^{3/4} + \sqrt{n} \}\,.
    \end{align*}
    Notice that $E$ not occuring means that the first ``return failure'' line is not executed. If none of these events happens, then we must have
    \[
    s < \tfrac{n}{2}, 
    \quad
    s + t \;\ge\; \tfrac{n}{2},
    \quad
    s \;\ge\; \tfrac{n}{2} - 2\,n^{3/4},
    \quad
    \text{and}
    \quad
    s + t < \tfrac{n}{2} + 2\,n^{3/4}.
    \]
    One can check that these conditions imply exactly
    \[
    \bigl(s < \tfrac{n}{2}\bigr)
    \quad\text{and}\quad
    \bigl(s + t \ge \tfrac{n}{2}\bigr)
    \quad\text{and}\quad
    t < 4\,n^{3/4},
    \]
    so the algorithm’s final check (line~\ref{alg:randmedian:check} of the pseudocode) succeeds. In that case, our pivot-based bracket \([\,p_\ell,\,p_r\,]\) indeed contains the median, and the algorithm returns the correct value.
    
    \paragraph{Probability of each bad event.}
    We next prove that \(\Prob{A},\Prob{B},\Prob{C},\Prob{D},\Prob{E}\) are each at most \(n^{-1/4}\). Observe:
    \begin{itemize}
    \item Event \(A\) means \(s \ge n/2\). Hence among the \(n/2\) smallest elements of \(a\) (i.e.\ the set \(S_{n/2}\)), at most \(\tfrac12\,n^{3/4} - \sqrt{n}\) of them are drawn into \(b\). In other words,
    \[
    X_{\,n/2} \;\;\le\;\; \tfrac12\,n^{3/4} \;-\; \sqrt{n}.
    \]
    But from \(\E{X_{\,n/2}} = (n/2)\,n^{-1/4} = \tfrac12\,n^{3/4}\) and \eqref{eq:fail}, we get 
    \(\Prob{X_{\,n/2} \le \tfrac12\,n^{3/4} - \sqrt{n}} \le n^{-1/4}\). Thus \(\Prob{A} \le n^{-1/4}\).
    
    \item Event \(B\) means \(s+t < n/2\). Equivalently, the set \(S_{n/2}\) has more than \(\tfrac12\,n^{3/4} + \sqrt{n}\) elements contained in \(b\). Hence
    \[
    X_{\,n/2} \;\;\ge\;\; \tfrac12\,n^{3/4} \;+\; \sqrt{n}.
    \]
    Again, from \(\E{X_{\,n/2}} = \tfrac12\,n^{3/4}\) and \eqref{eq:fail}, we get 
    \(\Prob{X_{\,n/2} \ge \tfrac12\,n^{3/4} + \sqrt{n}} \le n^{-1/4}\). So \(\Prob{B}\le n^{-1/4}\).
    
    \item Event \(C\) means \(s < n/2 - 2\,n^{3/4}\). In this case, the set \(S_{\,n/2 - 2\,n^{3/4}}\) must have at least \(\tfrac12\,n^{3/4} - \sqrt{n}\) of its elements in \(b\). Formally,
    \[
    X_{\,n/2 - 2\,n^{3/4}} \;\;\ge\;\; \tfrac12\,n^{3/4}-\sqrt{n}.
    \]
    Using \(\E{X_{\,n/2 - 2\,n^{3/4}}} = (n/2 - 2\,n^{3/4})\,n^{-1/4} = \frac{1}{2}n^{3/4} - 2\sqrt{n}\)  and \eqref{eq:fail} again shows 
    \(\Prob{C} \le n^{-1/4}\).
    
    \item Event \(D\) means \(s+t \ge n/2 + 2\,n^{3/4}\). Hence the set \(S_{\,n/2 + 2\,n^{3/4}}\) contains at most \(\tfrac12\,n^{3/4} + \sqrt{n}\) elements from \(b\). That is,
    \[
    X_{\,n/2 + 2\,n^{3/4}} 
    \;\;\le\;
    \tfrac12\,n^{3/4} \;+\; \sqrt{n}.
    \]
    A final application of \eqref{eq:fail} yields \(\Prob{D}\le n^{-1/4}\).

    \item A similar calculation involving $X_n$ shows that \(\Prob{E}\le n^{-1/4}\).    
\end{itemize}

    \paragraph{Conclusion.}
    By the union bound,
    \[
    \Prob{A \,\cup\, B \,\cup\, C \,\cup\, D \cup\, E}
    \;\le\;
    \Prob{A}+\Prob{B}+\Prob{C}+\Prob{D}+\Prob{E}
    \;\le\;
    5\,\times n^{-1/4}
    \]
    In other words, with probability at least \(1 - 5n^{-1/4}\), \emph{none} of these events occurs. When all five do not occur, the algorithm’s checks
    \[
    s < \lceil n/2\rceil,\quad
    s + t \ge \lceil n/2\rceil,\quad
    t < 4\,n^{3/4}
    \]
    are satisfied, so the algorithm terminates successfully with the correct median. Hence the overall success probability is at least \(1 - 5n^{-1/4}\), completing the proof.
    \end{proof}
    

\section*{Runtime Analysis}

Next, we establish that Algorithm~\ref{alg:randmedian} achieves linear time complexity \emph{with high probability}. Note that if the algorithm returns \emph{failure}, it may be retried or combined with a fallback procedure. Here, we focus on the typical execution that succeeds.

\begin{theorem}[Runtime]
\label{thm:runningtime}
Conditioned on the event that Algorithm~\ref{alg:randmedian} does not fail (an event which, by Theorem~\ref{thm:correctness}, holds with high probability), the total running time is \Oh*{n}.
\end{theorem}

\begin{proof}
We break the cost into four parts:
\begin{enumerate}
    \item \textbf{Sampling and Sorting \(b\).}
    We pick \(m \le n^{3/4} + \sqrt{n} \le 2 n^{3/4}\) elements randomly from \(a\) (conditioned on success). This is done in \Oh*{n} time. Sorting \(b\) of size \(m\) then takes
    \[
    \Oh*{m \Log*{m}} 
    =
    \Oh*{n^{3/4} \Log*{n^{3/4}}}.
    \]
    Since \(n^{3/4} \Log*{n^{3/4}}\) is asymptotically smaller than \(n\) for large \(n\), this component is \oh*{n}.

    \item \textbf{Computing Ranks and Sub-array \(T\).}
    We determine how many elements of \(a\) lie below \(p_\ell\), how many lie above \(p_r\), etc. This amounts to a single pass through \(a\), comparing each element of \(a\) to \(p_\ell\) and \(p_r\). Hence, it costs \Oh*{n} time.

    \item \textbf{Sorting \(T\).}
    By Algorithm~\ref{alg:randmedian}, we only proceed to sort \(T\) if \(t < 4n^{3/4}\). Thus,
    \[
    t \le 4n^{3/4}.
    \]
    Sorting \(T\) via a standard comparison-based algorithm costs
    \[
    \Oh*{|T|\Log*{|T}|}
    \le
    \Oh*{n^{3/4} \Log*{\bigl(n^{3/4}\bigr)}},
    \]
    which again is \oh*{n}.

    \item \textbf{Extracting the Median.}
    After sorting \(T\), we simply pick the element of \(T\) whose rank corresponds to \(\lceil n/2\rceil - s\). This is clearly an \Oh*{1} operation.
\end{enumerate}

Summing these components, the \emph{dominant term} is \Oh*{n} (from the pass to identify \(s\) and \(t\)). The sublinear sorting costs on \(b\) and on \(T\) both constitute \Oh*{n^{3/4}\Log*{n}}, which is strictly smaller than \(n\) for large \(n\). Consequently, the total running time is \Oh*{n}.
\end{proof}

In conclusion, the previous two theorems immediately prove~\Cref{theorem:mainTheorem}, our main result, which we restate here for convenience.
\mainTheoremCommand*

\section*{Discussion and Extensions}

Theorems~\ref{thm:correctness} and~\ref{thm:runningtime} jointly show that Algorithm~\ref{alg:randmedian} achieves \Oh*{n} time with high probability of success. In rare cases where the check fails, the algorithm returns \emph{failure} rather than risking an incorrect answer. One can either re-run the algorithm (the probability of repeated failures being very low) or combine it with a fallback deterministic linear-time median-finding routine such as the median-of-medians approach. The randomized algorithm typically attains lower hidden constants in practice than purely deterministic linear-time solutions, thereby offering an attractive balance of theoretical guarantees and real-world performance.

In the next chapter, we summarize our results and highlight possible directions for future work on randomized selection algorithms.


\chapter{Conclusion}
\label{ch:conclusion}

In this thesis, we presented a randomized selection algorithm for identifying the median of an unsorted array in expected linear time. Our method hinges on sampling a sublinear number of elements, sorting them to obtain pivots, and then “zooming in” on a small region of the array that is guaranteed, with high probability, to contain the true median. The crux of our analysis relied on Chebyshev’s inequality, which provided a principled way to bound the probability that our random pivots deviate significantly from the actual median’s rank in the sorted order.

\paragraph{Summary of Contributions.}
\begin{itemize}
    \item We introduced a randomized median-finding algorithm that leverages a sample of size roughly \( n^{3/4}\). Through sorting this sample and selecting pivots near its middle, the algorithm narrows down the search space to a sub-array whose size is no more than \(4n^{3/4}\).
    \item We proved a high-probability bound (Theorem~\ref{thm:correctness}) on the correctness of the algorithm. Specifically, with probability at least \(1 - \Oh*{n^{-1/4}}\), our pivots capture the true median, ensuring that the algorithm does not report \emph{failure}.
    \item We established a near-linear worst-case running time (Theorem~\ref{thm:runningtime}). The cost of sorting the small sample and, conditionally, the sub-array \(T\) both remain asymptotically lower than \Oh*{n}, leaving the dominant cost to be a single pass over the array to compute element ranks relative to the pivots. Hence, the overall algorithm runs in \Oh*{n} time with high probability.
    \item Beyond the immediate theoretical guarantees, we discussed how to combine our randomized approach with a deterministic “median-of-medians” fallback to handle rare failure events, ensuring correctness remains absolute. This hybrid strategy preserves the practical advantages of randomization while delivering a worst-case linear-time bound.
\end{itemize}

\paragraph{Limitations and Possible Extensions.}
While our use of Chebyshev’s inequality is conceptually simple and broadly applicable, tighter concentration results (e.g.\ Chernoff or Hoeffding bounds) could potentially yield smaller failure probabilities or allow an even smaller sample size. However, such refinements might require more specific assumptions about the distribution of the input data or the independence structure of the sampling process.

Another direction for future exploration is to extend the approach to streaming and dynamic settings. In scenarios where the dataset arrives incrementally, maintaining a small, representative sample to approximate the median could be achieved through reservoir sampling or related techniques. Additionally, one might investigate parallel or distributed variants of the algorithm, where the sampling and pivot-selection steps are split across multiple processors or machines.


\backmatter

\bibliographystyle{plain}
\bibliography{refs}

\appendix
\input{appendix}

\includepdf[pages={-}]{declaration-originality.pdf}

\end{document}
